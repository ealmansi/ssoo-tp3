\section{Análisis de los datos}

A continuación detallamos la implementación de las funciones para cada uno de los análisis pedidos:

\subsection{Punto 1}

Encontrar las doce películas mejor rankeadas de la colección de reseñas con al menos veinte reseñas.

\subsubsection{Descripción}

Función \textbf{map}:\\\\
En este paso emitimos como key al \emph{productId} del documento (que identifica a la película), y como \emph{value} un objeto con las siguientes propiedades:

\begin{description}
	\item[count:] cantidad de reseñas.
	\item[score:] puntaje otorgado a la película por el reviewer.
\end{description}

\begin{lstlisting}[frame=leftline]
  function() {
    key = this.productId;
    value = { 
      count: 1,                       // ocurrences
      score: parseInt(this.score,10)  // score
    };
    emit(key, value);
  }
\end{lstlisting}

Función \textbf{reduce}:\\\\
En este paso se acumula la cantidad total de reseñas en la propiedad \textbf{count} y la suma de todos los puntajes en la propiedad \textbf{score}:

\begin{lstlisting}[frame=leftline]
  function (key, values) {
    reducedValue = {
      count: 0,
      score: 0,
    };
    for (var i=0; i < values.length; i++){
      reducedValue.count += values[i].count;	// total ocurrences
      reducedValue.score += values[i].score;	// total score
    }
    return reducedValue;
  }
\end{lstlisting}

Función \textbf{finalize}:\\\\
Como queremos considerar sólo las películas con al menos veinte reseñas, hacemos uso de esta función con el siguiente código:\\

\begin{lstlisting}[frame=leftline]
  function (key, reducedValue) { 
    if(reducedValue.count >= 20){
      return reducedValue.score/reducedValue.count;	// average score
    }
    else{
      return 0;
    }
  }
\end{lstlisting}

Donde si la cantidad de reseñas es mayor o igual a 20, devolvemos el promedio de las mismas dividiendo el score total por la cantidad de reseñas, caso contrario devolvemos 0.

\subsubsection{Resultado}

Para la obtención de los datos pedidos modificamos el archivo \textit{runner.py} para que guarde la ejecución de \textbf{mapreduce} en una nueva colección llamada ``ej1'' y luego le pida a \textit{mongo} que ordene los resultados en forma decreciente respecto al valor obtenido y nos devuelva los primeros 12 resultados solamente.\\

\begin{lstlisting}[frame=leftline]
getattr(db,outcoll).aggregate([{"$sort":SON([("value",-1)])},
  {"$limit":12},{"$out":outcoll}])
coll = getattr(db,outcoll).find()
print "Los producId de las 12 peliculas mejor rankeadas en amazon son (",
for res in coll:
	print res['_id'],
print ")"
\end{lstlisting}

Los resultados obtenidos fueron los siguientes:
\begin{enumerate}
\item B0002NY7UY : Live in Concert [VHS]
\item B0007GAEXK : The Mole - The Complete First Season (2001)
\item B00000JSJ5 : All Creatures Great and Small, Series 2: Volumes 1-6 [VHS] (1990)
\item B0007Z4HAC : Salsa Crazy Presents: Learn to Salsa Dance, Intermediate Series, Volume 1 (2005)
\item B000AOEPU2 : WWE: Bret "Hitman" Hart - The Best There Is, The Best There Was, The Best There Ever Will Be (2005)
\item B00004YKS7 : Genghis Blues (1999)
\item B00004YKS6 : Genghis Blues [VHS] (1999)
\item B000MCIADA : A Reiki 1st, Aura and Chakra Attunement Performed (2007)
\item B008COIZHQ : Genghis Blues 1999
\item B006JN87UC : Transformers: Prime - Season One (Limited Edition) [Blu-ray] (2012)
\item B003YBGJ4S : WELL worked out with Tannis
\item B004LK24BI : 50/50 Cardio and Weights with Angie Gorr (2011)
\end{enumerate}

