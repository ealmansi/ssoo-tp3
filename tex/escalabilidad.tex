\section{Análisis de escalabilidad}

Para la base de datos de prueba (de unos $4GB$ aproximadamente), el análisis de las palabras más frecuentes ya requiere un tiempo de cómputo de varios minutos utilizando una única computadora; un crecimiento constante en el volúmen de datos solo puede empeorar la perfomance. Por esto mismo, para reducir los requerimientos de almacenamiento y mejorar la performance de una forma escalable, es necesario realizar el cómputo de forma distribuída.

El esquema \emph{MapReduce} se desarrolló específicamente para resolver este problema, permitiendo paralelizar de forma natural el procesamiento de los datos entre las múltiples computadoras de un cluster. En este caso en particular, si las reseñas de la base de datos se encuentran repartidas entre múltiples servidores, cada uno de ellos tiene menores requerimientos de almacenamiento, y puede procesar la función \textbf{map} sobre cada una de sus reseñas independientemente de las demás. Adicionalmente, cada servidor puede ejecutar la función \textbf{reduce} sobre los valores obtenidos, realizando reducciones parciales sobre los datos que tiene al alcance.

Esta técnica de distribución de los datos se conoce como \emph{Sharding}\footnote{http://docs.mongodb.org/manual/core/sharding-introduction/}. Es posible almacenar una base de datos a lo largo de múltiples servidores virtuales, utilizando un motor de base de datos como \emph{MongoDB} y una plataforma tipo \emph{IaaS} que provea flexibilidad cuando sea necesario escalar el volúmen de datos o mejorar la performance de las consultas.

\subsection{Amazon EC2}

\emph{Amazon EC2 (Elastic Compute Cloud)} fue de las primeras plataformas \emph{IaaS} en aparecer y se mantiene como una de las más avanzadas. Provee un enorme abanico de posibilidades para la creación, configuración y monitoreo de máquinas virtuales o \emph{AMI}s (\emph{Amazon Machine Instance}). Se comienza seleccionando un tipo de instancia según la cantidad de virtual cores, memoria, espacio en disco y velocidad del core que se necesite. Las instancias puede correr varios sistemas operativos, incluyendo Windows, Linux, etc, y se puede elegir entre instancias ya predefinidas o construir una propia. Cada instancia puede correrse en datacenters diferentes, ubicados en distintas regiones geográficas. Esto puede utilizarse para seleccionar la región con menor latencia o establecer un criterio de redundancia ante cualquier problema con una región específica.

Una ventaja adicional de \emph{Amazon EC2} es el hecho de que hay abundante bibliografía sobre cómo establecer bases de datos (utilizando \emph{MongoDB} por ejemplo) de forma escalable, utilizando máquinas virtuales con \emph{Linux}.\footnote{http://docs.mongodb.org/ecosystem/platforms/amazon-ec2/}

Una opción para mantener los datos de las reseñas y procesar las consultas en tiempos razonables podría ser mantener múltiples instancias medianas \emph{On Demand}\footnote{http://aws.amazon.com/ec2/pricing/}:

\begin{verbatim}
              vCPU  ECU Memory (GiB)  Instance Storage (GB) Linux/UNIX Usage
  m3.medium   1     3   3.75          1 x 4 SSD             $0.070 per Hour
\end{verbatim}

Las mismas se cobran solo por el tiempo en que son efectivamente utilizadas, permitiendo alternar la cantidad de instancias en función de los requerimientos de uso, así como extender el volúmen de datos cuando sea necesario.

Sin embargo, si se puede estimar con precisión el ritmo de crecimiento del volúmen de datos o nivel de uso (por ejemplo, un cierto porcentaje anual), una mejor opción sería tomar instancias por tiempo determinado. 

En el caso de tomar instancias medianas por períodos anuales\footnote{http://aws.amazon.com/ec2/pricing/}:

\begin{verbatim}
              Upfront Hourly         
  m3.medium   $110    $0.064 per Hour
\end{verbatim}

Esto reduce ampliamente los costos, pero a la vez permite incrementar el número de instancias al término del período.

\subsection{Linode}

Linode es un respetado proveedor de VPSs o virtual private servers con Linux. Ofrece diversos planes que se diferencian con respecto a la cantidad de RAM del servidor, la cantidad de cores asignados, los límites de espacio en disco y tranferencia de datos mensuales. La elección de cada plan conlleva un gasto mensual fijo, se usen o no los recursos disponibles en el plan contratado. 

Linode cuenta con 6 datacenters en 3 regiones (continentes) diferentes, pudiendo elegir el datacenter en la región que brinda los menores tiempos de latencia (proveen una página de testing para ello). También dispone de un panel de control a través del cual se puede manejar el ciclo de vida de las máquinas virtuales, y un API para realizar tareas administrativas en forma programática. Para crear una máquina virtual sólo debe elegirse la distribución requerida y se procede a instanciarla con el acceso a los recursos ofrecidos por el plan elegido.

Otros servicios que ofrecen son administración del DNS de cada nodo, monitoreo, backup, y balanceo de datos entre un cluster de nodos en caso de necesitar tales configuraciones (con diferentes características de configuración, tales como sticky sessions, chequeo dinámico de servidores activos, etc). Varios de estos servicios con costo aparte.
