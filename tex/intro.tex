\section{Introducción}

Con la motivación de estudiar el problema de almacenar y analizar grandes cantidades de datos, dentro del área que se conoce como \emph{Big Data}, contamos con una base de datos de más de medio millón de reseñas reales de la tienda virtual de \emph{Amazon}. Las reseñas son acerca de películas comercializadas en el sitio, habiendo sido escritas por usuarios reales; cada una contiene varios campos de información, entre ellos un identificador del producto sobre el que se hace referencia (\textbf{productId}), el puntaje que asignó el usuario a la película (\textbf{score}), un índice de valoración de la reseña según otros usuarios del sitio (\textbf{helpfulness}), o el texto completo con la opinión del autor (\textbf{text}).

En una primera etapa, realizamos un análisis sobre los datos con la finalidad de obtener la siguiente información: cuáles son las películas mejor rankeadas (según el \textbf{score} de las reseñas), cuáles son las palabras más utilizadas para opinar sobre los productos en función del puntaje que asignó el autor, y cuál es la relación entre la longitud de las reseñas y el índice de valoración que han recibido. Esto se realizó siguiendo el esquema \emph{MapReduce}, implementando las funciones \textbf{map}, \textbf{reduce} y opcionalmente \textbf{finalize} para cada una de las preguntas que se deseaba responder.

Finalmente, estudiamos la escalabilidad de la solución propuesta, de forma tal de poder almacenar una cantidad creciente de datos, manteniendo a su vez el tiempo requerido para el análisis de los mismos dentro de rangos aceptables.